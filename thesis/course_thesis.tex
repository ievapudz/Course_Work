\documentclass[12pt]{report}

\usepackage{indentfirst}
\usepackage{hyperref}
\usepackage{graphicx}

\setcounter{secnumdepth}{0}

\begin{document}
	\begin{center}
	    \vspace*{1cm}
	    \Large
	    VILNIUS UNIVERSITY
	    
	    \vspace*{4cm}
        \Large
        IEVA PUDŽIUVELYTĖ
	    
        \vspace*{2cm}
        \Large
        PROTEIN THERMOSTABILITY PREDICTION USING 
		SEQUENCE REPRESENTATIONS FROM PROTEIN 
		LANGUAGE MODELS 
        
        \vspace*{2cm}
        \large
        Course thesis
        
        \vspace*{2cm}
        \large
        Vilnius, 2022
        
	\end{center}
	
	\newpage

	\tableofcontents

	\newpage
	
	\Large
	\section{Introduction}

    \vspace*{1cm}
        
	\normalsize

	The variety of proteins is no less diverse than the variety of organisms. 
	Just as the latter set is divided into domains, there are different 
	attempts to classify proteins into distinct subsets. One way is to 
	consider the heat-resistance property of biological macromolecules, which 
	is an important trait for practical applications, for example, PCR.

	Earlier studies show that protein's sequence and structural properties 
	influence the thermostability of the macromolecule (Modarres et al. 2016). 
	Furthermore, one of the most recent achievements in the field of deep 
	learning are transformer architecture-based language models or, particularly, 
	protein language models that have not yet been used to classify proteins based on
	their thermostability. Therefore, it was decided to apply protein 
	representations from the protein language model to make inferences about 
	thermostability of the biological macromolecules.

	There are transformer architecture-based language models trained in an 
	unsupervised fashion to predict probabilities of elements in sequences 
	(Devlin et al. 2018). 
	Simultaneously, the process of training creates sequences' embeddings – the 
	real numbered vectors that represent semantic connections of language 
	components. These representations can be transferred as input to specific 
	application models trained using a supervised learning method to complete 
	the defined task, such as the classification problem. Unsurprisingly, the 
	transition between two types of learning has a name of 'transfer learning'. 
	This separation is practically useful because the computationally-heavy 
	task to train the language model can be excluded from the development of 
	the application model.

	Since proteins can be represented in aminoacid sequences assumed as a 
	particular language, there are protein language models – models trained on 
	protein sequences – that provide embeddings as output. The multi-dimensional 
	vectors are transferred for the application neural network as its input to 
	observe the results and decide whether the computed representations are 
	suitable to solve the specific biological task.

	This work presents a novel way to predict thermal stability of proteins. 
	The solution is a feed- forward neural network (FNN). To train the FNN, 
	the evolutionary scale model 1b (ESM-1b) (Rives et al. 2021) is used to 
	generate embeddings for proteins of organisms with annotated growth 
	temperatures (Engqvist 2018). The model 
	takes the generated embedding to predict the thermostability class of the 
	input protein.

	\newpage

	\section{Theory}

	The main objective of the work is to create a classifier that could 
	determine to which thermostability class the protein sequence belongs.

	Thermostability classes were created by setting the threshold of 
	$65\ ^\circ$C - proteins that are stable at temperatures strictly lower 
	than the given threshold should be assigned to class '0', whereas the 
	remaining ones compose the class '1'. 

	\subsection{ESM-1b embeddings}

	Due to the novelty of embeddings, a considerably good performance of
	protein language models, and a recently emerged availability of embeddings,
	it was decided to develop a neural network model that would take protein
	embeddings from ESM-1b model as input and give the thermostability class
	label as output.

	ESM-1b is one of evolutionary scale models trained by Facebook Research 
	(Rives et al. 2021). The model has 33 layers and 650 million parameters. 
	The model was trained in an 
	unsupervised fashion on UniRef50 dataset (March 28, 2018). In order to ensure
	determinism in the validation set, authors removed protein sequences that
	were longer than 1024 aminoacids. 

	The authors made a script to extract model's embeddings available in the 
	repository "Evolutionary Scale Modeling". The script allows to choose 
	from which model and layer embeddings will be taken, what embeddings 
	(mean, per aminoacid, or beginning of the sequence token) to keep. In the
	result of using the script, a 1280 dimensional vector for protein is generated.
	
	The fact that sequences longer than 1024 aminoacids were removed from the 
	validation dataset for ESM-1b model's training implies to the limitation of 
	model's embeddings, which cannot be generated for sequences longer than 
	1024 aminoacids. For this reason, various methods to get the most accurate 
	prediction for longer sequences were tried.

	\newpage

	\section{Methods}

	Generally, the workflow consisted of the following steps:

	\begin{enumerate}
		\item Collecting sets of sequences
		\item Calculating ESM-1b embeddings for the sets of sequences
		\item Processing the set of generated embeddings
		\item Visualising ESM-1b embeddings
		\item Training and validating the neural network model of the chosen architecture 
		\item Testing the trained neural network model
		\item Presenting the results of the model
	\end{enumerate}

	The workflow will be reviewed in terms of these points.

	\subsection{Collecting data}

	Initially, the steps prior to the neural network model construction 
	were done using a small dataset that was composed of proteomes of two 
	organisms: a mesophilic bacteria \textit{Escherichia coli} 
	(\href{https://www.uniprot.org/proteomes/UP000000625}{UP000000625}) and a 
	thermophilic archaeon \textit{Sulfolobus solfataricus} 
	(\href{https://www.uniprot.org/proteomes/UP000001974}{UP000001974}). The 
	growth temperature of \textit{E. coli} is $37\ ^\circ$C (Jang et al. 2017) 
	and $80\ ^\circ$C for \textit{S. solfataricus} (Zaparty et al. 2010). This 
	dataset was named '001' and used only for embeddings visualisation.

	The visualisation of '001' stimulated to check whether embeddings are 
	distinguished by the thermostability property or the life domain has a
	significant impact to the data clusterisation. Subsequently, '002' dataset 
	was a collection of 2 mesophilic archaea and 2 thermophilic bacteria
	proteomes.

	\vspace*{0.5cm}

	Mesophilic archaea:

	\begin{itemize}
		\item \textit{Methanobrevibacter oralis} 
		(\href{https://www.uniprot.org/proteomes/UP000077428}{UP000077428})
		\item \textit{Nitrosopumilus maritimus} strain SCM1 (\href{https://www.uniprot.org/proteomes/UP000000792}{UP000000792})
	\end{itemize}

	Thermophilic bacteria:

	\begin{itemize}
		\item \textit{Aquifex aeolicus} (strain VF5)
		(\href{https://www.uniprot.org/proteomes/UP000000798}{UP000000798})
		\item \textit{Thermotoga maritima} 
		(strain ATCC 43589 / DSM 3109 / JCM 10099 / NBRC 100826 / MSB8) 
		(\href{https://www.uniprot.org/proteomes/UP000008183}{UP000008183})
	\end{itemize}

	After receiving the results of the first neural network model, it was decided
	to train and test the model on the first real dataset. '003' dataset was a subset
	of the dataset of 21458 annotated organisms (Engqvist 2018). 
	
	Requirements for '003' dataset were: the dataset needed to be balanced and a
	single taxonomy identifier can be apparent only in either training, validation,
	or testing dataset.

	\begin{table}[h!]
		\caption{Numbers of different proteomes that compose training, 
		validation, and testing datasets}
		\label{table:1} 
		\vspace{0.2cm}
		\centering
		\begin{tabular}{ c c c }
			 & Class '0' & Class '1' \\
			Training & 32 & 77 \\ 
			Validation & 8 & 17 \\
			Testing & 11 & 17   
		\end{tabular}
	\end{table}

	\begin{table}[h!]
		\caption{Numbers of aminoacid sequences that compose training, 
		validation, and testing datasets}
		\label{table:2} 
		\vspace{0.2cm}
		\centering
		\begin{tabular}{ c c c }
			 & Class '0' & Class '1' \\
			Training & 145128 & 143868 \\ 
			Validation & 33204 & 32616 \\
			Testing & 38263 & 36245   
		\end{tabular}
	\end{table}

	\newpage

	\section{Results}

	\dots

	\section{Conclusions}

	\dots

	\nocite{*}
	
	\normalsize

\bibliography{references} 
\bibliographystyle{ieeetr}

\end{document}
